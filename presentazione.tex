\documentclass{beamer}

\usetheme{Padova}

\newsavebox{\authbox}
\sbox{\authbox}{
	\centering
	\begin{minipage}{0.45\linewidth}
		\centering\normalsize
		\textcolor{white}{Candidato} \par
		\textcolor{white}{Daniel Rossi}
	\end{minipage}
	\hfill
	\begin{minipage}{0.45\linewidth}
		\centering\normalsize
		\textcolor{white}{Relatore} \par
		\textcolor{white}{prof. Luigi De Giovanni}
	\end{minipage}
}
\newsavebox{\datebox}
\sbox{\datebox}{
	
	\begin{minipage}{0.45\linewidth}
		\setlength{\parindent}{0.25cm}
		\textcolor{white}{\indent 18 Dicembre 2018}
	\end{minipage}
}
\title{\small{Esame di Laurea in Informatica}}
\subtitle{\Large{Implementazione di modelli di programmazione matematica per problemi di bin packing}}
\author[Daniel Rossi]{%
	\usebox{\authbox}
}
\institute{\footnotesize{Dipartimento di Matematica ''Tullio Levi Civita''}}
\date{%18 Dicembre 2018
	\usebox{\datebox}
}


\begin{document}

%\begin{frame}
%\titlepage
%\end{frame}
\maketitle
.
.
.
.

\section{Introduzione}

\begin{frame}{L'azienda}
	\begin{minipage}[c]{0.45\textwidth}
		\includegraphics[width=1\linewidth]{figures/logo-transcel}
	\end{minipage}
	\hfill
	\begin{minipage}[c]{0.45\textwidth}
		\begin{itemize}
			\item tecnica groupage
			\item merci ADR
			\item servizio deposito
		\end{itemize}
	\end{minipage}
			
	\vspace{3.0em}
	\large{\uppercase{Software supporto decisionale}}
	\begin{itemize}
		\item operatori meno esperti
		\item aumento della produttivit\`a
		\item informazioni sullo stato dei trasporti
		\item stima di costi e profitti
	\end{itemize}
\end{frame}

\begin{frame}{Tool aziendale}
	L'azienda ha sviluppato un'euristica per l'ottimizzazione dello spazio occupato dalle merci nel cassone del camion
	\begin{figure}[H]
		\begin{center} \includegraphics[width=1\linewidth]{figures/container_arrows}
		\end{center}
	\end{figure}
\end{frame}

\section{Progetto}
\begin{frame}{Proposta di stage}
	\begin{alertblock}{Scopo}
		Lo scopo dello stage \`e  quello di realizzare dei modelli di programmazione lineare per la risoluzione dello \\ \textbf{Strip Packing Problem} da usare per valutare l'euristica aziendale
	\end{alertblock}
	\begin{itemize}
		\item \textbf{2D}: versione 2D
		\item \textbf{2DR}: versione 2D con rotazione
		\item \textbf{3D}: versione 3D con rotazione e sovrapposizione
	\end{itemize}
\end{frame}

\begin{frame}{Packing Problem}
	Insieme $I = \{1,\dots,n\}$ di oggetti aventi dimensioni $w_{i}$, $d_{i}$ e $h_{i}$	
																		
	Insieme $J = \{1,\dots,m\}$ di contenitori di dimensione $W$, $D$ e $H$
																
	Per ipotesi $w_{i} \leq W$, $d_{i} \leq D$ e $h_{i} \leq H$
	\vspace{.5em}
	\begin{alertblock}{Obiettivo Strip Packing}
		Minimizzare i metri lineari occupati dagli oggetti dell'insieme $I$ rispetto la profondit\`a del contenitore
	\end{alertblock}	
	Obiettivo Bin Packing \`e invece minimizzare il numero di contenitori $J$ che riescano a contenere tutti gli oggetti dell'insieme $I$
\end{frame}


\begin{frame}{Base dei modelli matematici \scriptsize{(Blum e Schmid, 2013)}}
	\begin{align*}
		& & &\underset{}{\text{min D}} \\
		& \text{s.t.} \\
		  &   &   & l_{ij} + l_{ji} + b_{ij} + b_{ji} \geq 1      & i < j    &   & i,j \in I \\
		  &   &   & y_i - y_j + M_d b_{ij} \leq M_d - d_i         &          &   & i,j \in I \\
		  &   &   & x_i - x_j + M_w l_{ij} \leq M_w - w_i         &          &   & i,j \in I \\
		  &   &   & x_i + w_i \leq W                              &          &   & i \in I   \\
		  &   &   & y_i + d_i \leq D                              &          &   & i \in I   \\
		  &   &   & b_{ij}, l_{ij} \in \{0,1\}                    & i \neq j &   & i,j \in I \\
		  &   &   & x_{i}, y_{i}, w_{i}, d_{i} \in \mathbb{R}^{+} &          &   & i \in I   
	\end{align*}
	\scriptsize{Articolo: Solving the 2D bin packing problem by means of a hybrid evolutionary algorithm, C. Blum and V. Schmid, 2013}
\end{frame}

\begin{frame}{Tecnologie}
	\vspace{.5em}
	\begin{minipage}[c]{0.2\textwidth}
		\begin{figure}
			\centering
			\includegraphics[width=1.8cm]{figures/coin_banner}
			\vspace{.5em}
			\includegraphics[width=1.8cm]{figures/python}
			\vspace{.5em}
			\includegraphics[width=1.8cm]{figures/jupyter}
		\end{figure}
	\end{minipage}
	\hfill
	\begin{minipage}[c]{0.6\textwidth}
		\begin{figure}
			\centering
			\includegraphics[width=5cm]{figures/matplotlib-1}
			\vspace{.5em}
			\includegraphics[width=5cm]{figures/pandas_logo}
			\vspace{.5em}
			\includegraphics[width=4cm]{figures/google_or_tools}
		\end{figure}
	\end{minipage}		
\end{frame}

\begin{frame}{Modello 2D e 2DR}
												
	\begin{minipage}[c]{0.45\textwidth}
		\textbf{Modello 2D}:\\Punto di partenza
	\end{minipage}
	\hfill
	\begin{minipage}[c]{0.45\textwidth}
		\includegraphics[width=1\linewidth]{figures/general2D}
	\end{minipage}
										
	\begin{minipage}[c]{0.45\textwidth}
		\textbf{Modello 2DR}:\\Possibilit\`a di sfruttare meglio gli spazi
	\end{minipage}
	\hfill
	\begin{minipage}[c]{0.45\textwidth}
		\includegraphics[width=1\linewidth]{figures/general2DR}
	\end{minipage}
\end{frame}

\begin{frame}{Modello 3D}
	\begin{minipage}[c]{0.45\textwidth}
		\textbf{Stabilit\`a sovrapposizione}:\\ Semplificata, ciascun oggetto potr\`a poggiare solo di un altro
	\end{minipage}
	\hfill
	\begin{minipage}[c]{0.45\textwidth}
		\includegraphics[width=1\linewidth]{figures/3dg}
	\end{minipage}
			
	\begin{minipage}[c]{0.45\textwidth}
		\textbf{Oggetti stackable}:\\ Alcuni oggetti potrebbero non essere sovrapponibili
	\end{minipage}
	\hfill
	\begin{minipage}[c]{0.45\textwidth}
		\includegraphics[width=1\linewidth]{figures/3d}
	\end{minipage}
\end{frame}

\begin{frame}{Modello 2DRS}
											
	\begin{minipage}[c]{0.45\textwidth}
		\textbf{Vie di scarico}:\\Deve essere presente almeno una via di scarico per ciascun pacco tra:
		\begin{itemize}
			\item $\alpha$: scarico da sinistra
			\item $\beta$: scarico da destra
			\item $\gamma$: scarico frontale
		\end{itemize}
	\end{minipage}
	\hfill
	\begin{minipage}[c]{0.45\textwidth}
		\includegraphics[width=1\linewidth]{figures/abg_2drs}
	\end{minipage}
																												
	\begin{minipage}[c]{0.45\textwidth}
		\textbf{Stabilit\`a della sequenza}:\\ Ogni oggetto non pu\`o trovarsi isolato dal resto delle merci durante il viaggio
	\end{minipage}
	\hfill
	\begin{minipage}[c]{0.45\textwidth}
		\includegraphics[width=1\linewidth]{figures/2d3d}
	\end{minipage}
\end{frame}

\begin{frame}{Test computazionale}
	\begin{minipage}[c]{0.45\textwidth}
		\textbf{Istanza}: insieme formato dai pacchi da disporre nel contenitore
		\vspace{.5cm}
												
		\textbf{Gruppo di istanze}: insieme di 100 istanze accomunate tra loro dalle loro dimensioni, identificati da un $Id$\\
												
		\textbf{Time Limit}: 300 secondi\\
												
		Soluzioni:
		\begin{itemize}
			\item ottime
			\item best bound
		\end{itemize}
	\end{minipage}
	\hfill
	\begin{minipage}[c]{0.45\textwidth}
		\begin{center}
			\begin{tabular}{c|c|c|c|c}
				$Id$ & $w_a$ & $w_b$ & $d_a$ & $d_b$ \\
				\hline	
				0    & 0.5   & 2.45  & 0.5   & 2.45  \\
				1    & 0.5   & 1.50  & 0.5   & 4.00  \\
				2    & 1.5   & 2.45  & 0.5   & 4.00  \\
				3    & 0.5   & 1.50  & 3.0   & 4.00  \\
				4    & 1.5   & 2.45  & 3.0   & 4.00  \\
				5    & 0.1   & 1.00  & 0.1   & 1.00  \\
				6    & 0.1   & 1.00  & 3.0   & 4.00  \\
				7    & 2.0   & 2.45  & 3.0   & 4.00  \\
				8    & 2.0   & 2.45  & 2.0   & 2.45  \\
				9    & 0.1   & 1.00  & 0.1   & 4.00  \\
			\end{tabular}
		\end{center}
		width: [$w_a$,$w_b$], depth: [$d_a$,$d_b$]
	\end{minipage}
\end{frame}

\begin{frame}{Sviluppo funzionalit\`a euristica}
	\begin{itemize}
		\item \textbf{CheckFeasible}: utilizzata per controllare che non vi fossero pacchi intersecati tra loro
		\item \textbf{CheckSequence}: realizzata in due versioni:
			  \begin{itemize}
				  \item \textbf{2D}: considerando oggetti alla stessa altezza, controlla ci sia una via di scarico
				  \item \textbf{3D}: considerando oggetti su pi\`u livelli, controlla ci fosse una via di scarico e che non vi siano altri pacchi sopra di esso
			  \end{itemize}
		\item \textbf{CheckStable}: utilizzata per controllare la stabilit\`a degli oggetti
		\item \textbf{ProjectionCommonArea}: utilizzata per verificare se due parallelepipedi avessero intersezioni tra le loro proiezioni rispetto ad una coppia di assi come altezza e profondit\`a.	
	\end{itemize}
\end{frame}

\begin{frame}{Risultati 2DR}
	\begin{minipage}{0.49\textwidth}
		\centering
		\textbf{Ottime}
						
		\vspace{.5em}
		\begin{tabular}{c|c|c|c|c}
			$Id$ & \#ist & $\epsilon_r$ & $\epsilon_a$ & Time  \\
			\hline
			0    & 64    & 3.89         & 0.23         & 40.95 \\
			1    & 73    & 11.90        & 0.81         & 31.51 \\
			2    & 76    & 0.94         & 0.10         & 19.76 \\
			3    & 84    & 12.29        & 1.26         & 19.79 \\
			4    & 75    & 0.00         & 0.00         & 27.69 \\
			5    & 73    & 14.17        & 0.11         & 12.58 \\
			6    & 78    & 6.60         & 0.47         & 20.95 \\
			7    & 76    & 0.00         & 0.00         & 36.62 \\
			8    & 81    & 0.00         & 0.00         & 23.70 \\
			9    & 81    & 10.34        & 0.45         & 10.60 \\
		\end{tabular}
	\end{minipage}
	\begin{minipage}{0.49\textwidth}
		\centering
		\textbf{Best bound}
						
		\vspace{.5em}
		\begin{tabular}{c|c|c|c}
			$Id$ & \#ist & $\epsilon_r$ & $\epsilon_a$ \\
			\hline
			0    & 36    & 7.35         & 0.59         \\
			1    & 27    & 15.98        & 1.48         \\
			2    & 24    & 0.91         & 0.17         \\
			3    & 16    & 17.26        & 2.62         \\
			4    & 25    & 0.00         & 0.00         \\
			5    & 27    & 16.16        & 0.21         \\
			6    & 22    & 19.40        & 1.70         \\
			7    & 24    & 0.00         & 0.00         \\
			8    & 19    & 0.00         & 0.00         \\
			9    & 19    & 20.58        & 1.10         \\
		\end{tabular}
	\end{minipage}
	\vspace{.5em}
	\begin{itemize}
		\item $\epsilon_a = Obj_h - Obj_m$
		\item $\epsilon_r = \frac{\epsilon_a}{Obj_m} \cdot 100$
	\end{itemize}	
\end{frame}
\section{Raggiungimento degli obiettivi}
\begin{frame}{Raggiungimento degli obiettivi}
	I 3 periodi di sviluppo prevedevano i seguenti obiettivi:
	\begin{itemize}
		\item analisi costraints
		\item prototipazione
		\item integrazione nel sistema
		\item testing e confronto	
	\end{itemize}
	
	Due variazioni nel corso dello stage:
	\begin{itemize}
		\item Confronto con l'euristica eseguito dopo la realizzazione dei modelli
		\item Nuove funzioni di verifica dell'euristica sviluppate durante il testing dei modelli
		\item Realizzazione nuovo modello non preventivato
	\end{itemize}
\end{frame}
\section{Consuntivo}
\begin{frame}{Consuntivo}
	In generale i periodi di sviluppo hanno richiesto:
	\begin{center}
		\begin{tabular}{|c|l|c|c|}
			\hline
			\textbf{\#} & \textbf{Obiettivi} & \textbf{Ore previste} & \textbf{Ore effettive} \\\hline
			1           & Modello 2D               & 96                    & 64                     \\\hline
			2           & Modello 2DR              & 64                    & 64                     \\\hline
			3           & Modello 2DRS             & 0                     & 72                     \\\hline
			4           & Modello 3D               & 64                    & 72                     \\\hline
			5           & Confronto con euristica  & 24                    & 16                     \\\hline
		\end{tabular}
	\end{center}
	Cause cambiamenti tempistiche:
	\begin{itemize}
		\item Formazione pi\`u breve
		\item Prototipazione modelli 2D e 2DR pi\`u rapida
		\item Testing e sviluppo funzionalit\`a euristica in parallelo
	\end{itemize}
\end{frame}
\section{Conclusioni}
\begin{frame}{Conclusioni}
	\begin{enumerate}
		\item Tutti gli obiettivi sono stati completati
		\item Realizzazione modello aggiuntivo
		\item Valutazione positiva dei risultati forniti
		\item Importanti conoscenze apprese
	\end{enumerate}
				
	\vspace{.5em}
	Sviluppi futuri:
	\begin{itemize}
		\item Archiviazione soluzioni modelli
		\item Stabilit\`a sequenza
		\item Stabilit\`a degli oggetti
		\item Sequenza di scarico multilivello
		\item Unione modelli
	\end{itemize}
\end{frame}

\end{document}
